\chapter{Literature Review}
The field of Artificial Intelligence in healthcare is rapidly evolving, with significant advances in: (i) predictive diagnostic models \cite{optimizing_ai_sepsis_2024}, (ii) Multimodal data integration, and (iii) Explainable AI techniques. 

\section{AI in Healthcare}

\subsection{Surveys}
Here are some survey \& position papers. Rajpurkar et al. \cite{moor2023GMAI} advocate for \textit{generalist medical AI}, i.e. models which are trained on large, unlabelled, diverse datasets with self-supervision, can flexibly ingest different modalities (e.g. imaging, EHR, genomics) and produce expressive outputs (e.g. free-text explanations, spoken recommendations). They argue that such models will be capable of carrying out a diverse set of tasks using very little or no task-specific labelled data.

\todo[inline]{Add surveys on sepsis, etc.}


\subsection{Healthcare Applications}
\textbf{Sepsis} \cite{singer2016sepsis3} is a life-threatening organ dysfunction caused by a dysregulated host response to infection. It is a leading cause of morbidity and mortality in hospitals, with an estimated 11 million deaths annually \cite{rudd2020global}. 

\textbf{Cardiac Arrest} is a critical condition that requires immediate medical intervention. It occurs when the heart stops beating effectively, leading to a lack of blood flow to vital organs.  There are two types of cardiac arrest: out-of-hospital cardiac arrest (OHCA) and in-hospital cardiac arrest (IHCA). OHCA survival rate to discharge is 10-12\%, while IHCA survival rate is 20-25\% \cite{andersen2019cardiac}. 80\% of presenting rhythms are non-shockable, meaning that defibrillation is not an option, i.e. that early detection is the best fix. The incidence is 9-10 per 1000 admissions \cite{andersen2019cardiac}.

\textbf{Breast Cancer} affects 1 in 8 women. They use recurrence gene assays (Breast Cancer Index, JCO). 



\section{Explainable AI}
\todo[inline]{read the LIME, SHApe papers, and add a summary here.}

\section{Multimodal AI}
\todo[inline]{read the papers on multimodal AI, and add a summary here.}

RQ: how do we compress exomic analysis? 
Videos belong in low-dimensional space, so do sequences also?

Dimension-reduction with phenotypes! 

HeLM
HAIM 