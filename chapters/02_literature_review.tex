\chapter{Literature Review}
The field of Artificial Intelligence in healthcare is rapidly evolving, with significant advances in: (i) predictive diagnostic models \cite{optimizing_ai_sepsis_2024}, (ii) Multimodal data integration, and (iii) Explainable AI techniques. 

\section{AI in Healthcare}

\subsection{Surveys}
Here are some survey \& position papers. Rajpurkar et al. \cite{moor2023GMAI} advocate for \textit{generalist medical AI}, i.e. models which are trained on large, unlabelled, diverse datasets with self-supervision, can flexibly ingest different modalities (e.g. imaging, EHR, genomics) and produce expressive outputs (e.g. free-text explanations, spoken recommendations). They argue that such models will be capable of carrying out a diverse set of tasks using very little or no task-specific labelled data.

\todo[inline]{Add surveys on sepsis, etc.}


\subsection{Healthcare Applications}
\label{sec:healthcare_applications}

\subsubsection{Sepsis}
\label{sec:sepsis}
\textbf{Sepsis}  is a life-threatening organ dysfunction caused by a dysregulated host response to infection \cite{singer2016sepsis3}. It is a leading cause of morbidity and mortality in hospitals, with an estimated 11 million deaths annually \cite{rudd2020global}. Organ dysfunction can be identified as an acute change in total SOFA score
$\geq 2$ points consequent to the infection, Fig \ref{fig:sofa}. Prior sepsis definitions used two of 4 SIRS criteria, which are not specific to sepsis and can lead to overdiagnosis(patients who has SIRS but not sepsis).


% insert inline SOFA figure in images/sofa.png
\begin{figure}[h]
    \centering
    \includegraphics[width=0.5\textwidth]{images/sofa.png}
    \caption{SOFA score criteria for organ dysfunction.}
    \label{fig:sofa}
\end{figure}

The SOFA score is a clinical score used to track a patient's status during their stay in an ICU. It is used to determine the extent of a person's organ function or rate of failure. The SOFA score is based on the following six organ systems: respiratory, coagulation, liver, cardiovascular, renal, and neurological. Each system is assigned a score from 0 to 4, with higher scores indicating more severe dysfunction. The total SOFA score is the sum of the individual scores for each organ system.

\textbf{ML-Driven Sepsis Detection} \\ 
Surveys: Islam et al. \cite{islam2023sepsis} does a meta-study. Moor et al. 


Papers
\textbf{2015-2017}: ?? 
\textbf{2017-2020}: TREWS \cite{adams2022trews} 
\textbf{2020-2022}: ??
\textbf{2022-}: \cite{optimizing_ai_sepsis_2024}


\subsubsection{Cardiac Arrest}
\label{sec:cardiac_arrest}
\textbf{Cardiac Arrest} is a critical condition that requires immediate medical intervention. It occurs when the heart stops beating effectively, leading to a lack of blood flow to vital organs.  There are two types of cardiac arrest: out-of-hospital cardiac arrest (OHCA) and in-hospital cardiac arrest (IHCA). OHCA survival rate to discharge is 10-12\%, while IHCA survival rate is 20-25\% \cite{andersen2019cardiac}. 80\% of presenting rhythms are non-shockable, meaning that defibrillation is not an option, i.e. that early detection is the best fix. The incidence is 9-10 per 1000 admissions \cite{andersen2019cardiac}.

\textbf{Breast Cancer} affects 1 in 8 women. They use recurrence gene assays (Breast Cancer Index, JCO). 



\section{Explainable AI}
\todo[inline]{read the LIME, SHApe papers, and add a summary here.}

\section{Multimodal AI}
\todo[inline]{read the papers on multimodal AI, and add a summary here.}

RQ: how do we compress exomic analysis? 
Videos belong in low-dimensional space, so do sequences also?

Dimension-reduction with phenotypes! 

HeLM
HAIM 